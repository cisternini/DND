\documentclass{article}
\usepackage{graphicx}
\usepackage{hyperref}
\usepackage{imakeidx}
\usepackage{rotating}
\usepackage{multirow}
\usepackage{array}
\usepackage[export]{adjustbox}
\usepackage{geometry}
\begin{document}
\section{Inizio Avventura}

\textbf{obiettivo Iniziale}: Muoversi verso Neverwinter in cerca di nuove datori di lavoro.
L’avventura ha inizio con il Gruppo che si avvicina alla Taverna “Nido del Grifone”, i proprietari sono indaffarati con una consegna per uno dei loro ospiti che ha inviato i propri bagagli in anticipo.
Charldeth e Lindlow Askew sono fratello e sorelle e gestiscono Il Nido del Grifone. 
\paragraph{Charldeth Askew}
Nano, Maschio, Cittadino, Proprietario
Charldeth è un vecchio tizio paffuto e rubicondo che ama un po' troppo la propria birra. Un torace pesante farà rotolare facilmente il nano brillo sulla schiena, ma è un tipo allegro, sempre pronto a ridere di se stesso – e degli altri – e mai uno che chiede aiuto o ammette di essere davvero nei guai. È apertamente generoso e ricompenserà qualsiasi assistenza con un pasto gratuito e bevande in casa per il gruppo
Charldeth si lamenta costantemente della sorella troppo premurosa, ma è abbastanza chiaro che tiene a lei nel profondo.

\paragraph{Lindlow Askew }
Nano, Femmina, Cittadino, Proprietario
Lindlow è forte e capace nonostante i suoi anni avanzano. Ha trascorso molti anni come primo ufficiale di una nave prima di ritirarsi per prendersi cura di suo fratello e impedirgli di mandare a rotoli dell'azienda di famiglia.
A volte sembra dimenticare che non è ancora a bordo della nave e abbaia ordini agli ospiti come se fossero il suo equipaggio. Odia vedere la gente inattiva quando c'è del lavoro da fare, ma offre sempre una caraffa di rum gratuita a tutti coloro che si danno da fare per aiutare.
Lindlow si dispera ad alta voce per suo fratello, ma è ovvio che ha un grande affetto per lui nel profondo.
\paragraph{Higustus Caltraire} Elfo, maschio, cittadino comune
Higustus è un uomo di poche parole. È cresciuto nell'orfanotrofio locale dove ha inventato la storia che i suoi genitori erano nobili che un giorno lo avrebbero mandato a chiamarlo e lo avrebbero benedetto con un'enorme eredità. Ha ripetuto la bugia così spesso che ora sembra crederci lui stesso. Di conseguenza, ha un'opinione estremamente esagerata di sé stesso, considerando la maggior parte delle persone al di sotto di lui.
È infastidito dal dover abbassarsi ad associarsi con la "marmaglia" come feste avventurose e lascerà che Delano e Tyrk parlino per la maggior parte.

\paragraph{Tyrk Goodwater} Maschio, nano, cittadino comune
Tyrk è molto vocale ma non è il membro più abile del gruppo. Il suo contributo alle loro imprese tende ad essere forza bruta e intimidazione. Ha il pugno stretto e si risente di dover pagare per qualsiasi cosa, mai. Di certo non è contento del suggerimento di Delano di perdere deliberatamente qualche giro di carte per conquistare questi sconosciuti... ma poi vuole sbarazzarsi di quel maledetto formaggio.
Tyrk finge un recente infortunio alla schiena come motivo per cui non possono portare il formaggio fino al castello di Roquefort da soli. Ogni volta che si ricorda (o quando Delano lo prende a calci sotto il tavolo) si lamenta e si lamenta a gran voce. Se qualcuno si offre di curarlo, si agita e si confonde e chiede aiuto a Delano.



\paragraph{Delano Fisk}Umano, maschio, cittadino comune
Delano si presenta come affascinante e disinvolto. È una persona scaltra e calcolatrice e cerca di prendere rapidamente la misura di ogni personaggio, decidendo il modo migliore per conquistare la loro fiducia e fiducia.
Perderà allegramente a carte, comprerà da bere, finge di interessarsi a racconti di eroismo, adula l'ego e fornisce liberamente informazioni sull'area o qualsiasi altra cosa a cui i personaggi sono interessati.
\paragraph{Scena}\textit{Mentre ti avvicini al Nido del Grifone, la tua strada viene bloccata da un vecchio nano tondo dalla faccia rossa che lotta con un carro carico di casse di legno. Con un ultimo OH ISSA! Il nano cade all'indietro sul terreno di fronte a te con uno dei forzieri sopra di lui. Sbattendo le palpebre, ti guarda e sorride a denti stretti attraverso la sua barba rossa e ingrigita: "Ah, non preoccupatevi di me, sarò lì per sistemarvi tra un minuto... ho solo bisogno di... ehm.... far spostare questo e sarò subito con te...}
\begin{itemize}

\item Se il gruppo decide di aiutare il padrone di casa, uno dei bauli è in realtà un Mimic Incinta (BR) che emette 5 Giovani Mimic (TCOE) sotto forma di feroci monete che mordono. Inoltre, 5 Giovani Mimic si nascondono nei vestiti e nei bagagli dei personaggi senza che se ne accorgano. Lancia il dado suggerito oppure decidi tu in base alle dimensioni del gruppo.
\item Se il gruppo decide di non aiutare, può aggirare il blocco con difficoltà e poi incontrare l'altro padrone di casa all'interno che può offrire loro una stanza e un rinfresco.
\end{itemize}
Una volta sistematisi all'interno della locanda i tre ladri di formaggio, Delano, Higustus e Tyrk, si avvicinano a loro chiedendo loro se vogliono unirsi a loro per una partita a carte. Se i personaggi rifiutano, si siedono comunque e Delano inizia una conversazione.
Difficilmente il gruppo accetterà di prendere il formaggio e infatti l'avventura sarà più interessante e divertente se si rifiutano! Ma se sono d'accordo, va bene lo stesso.
Una volta che hanno la fiducia del gruppo, cercano di convincerli a portare il formaggio a Château Roquefort. Nel caso molto probabile che i personaggi si rifiutino, gli imbroglioni provano qualche tattica in più e poi si arrendono, spostandosi dove due Tabaxi sono seduti dall'altra parte della stanza e cercano invece di persuaderli.
Tuttavia, i Chwinga (ROTFM) a questo punto hanno preso in giro gli avventurieri e trasferiranno la loro casa di formaggio in uno dei sacchi del gruppo in un momento in cui sono certi che non se ne accorgeranno per un po'.

\section{Capitolo 2}
Una volta che il gruppo si è lasciato alle spalle l'osteria e il paese, se non accettano di prendere il formaggio allora si accorgono del formaggio in mezzo a loro. Che il gruppo abbia preso il formaggio volontariamente o meno, diventa subito un fastidio: borbottare, canticchiare, sussurrare, ridacchiare, gridare insulti casuali e cantare canzoni offensive.
In un primo momento i PG dovrebbero avere l'impressione che i commenti offensivi e le rime ingiuriose provengano da altri membri del gruppo, ma alla fine possono dedurre che l'assalto provenga dal formaggio.
Qualsiasi tentativo di sbarazzarsene fallisce. Riappare semplicemente poche miglia dopo nello zaino di qualcun altro.
Qualsiasi tentativo di discernere la magia o rilevare la presenza del Chwinga viene respinto e accolto con risate insolenti e commenti beffardi dal ring.
Tutti i tentativi di distruggere il formaggio sono respinti dagli incantesimi protettivi che i Chwinga hanno messo nella loro nuova casa. Questi ultimi per tutta la durata dei Chwinga sono all'interno del formaggio. Se qualcuno tenta di distruggere il formaggio con un attacco banale o magico, tira sulla tabella dei colpi goffi formaggiosi qui sotto.

\textbf{Attacco Non Magico}
\begin{itemize}
    \item  Attacchi la cagliata sporca con una ferocia così mal orchestrata che la manchi completamente e la tua arma rimane saldamente conficcata nel terreno, richiedendo una prova di forza con CD15 per rimuoverla.
    \item Ti muovi selvaggiamente nella tua passione per sbarazzarti del formaggio demoniaco. Manchi completamente e colpisci invece la persona più vicina. (Questo potrebbe essere il PG più vicino o, a discrezione del DM, un NPG di passaggio.)
    \item La tua vista si offusca in una nebbia rossa di rabbia e colpisci ciecamente il ripugnante latticino, mancando di poco la sua superficie liscia e innocua e ferendo invece brutalmente il tuo stesso piede.
    \item Alimentato da un odio infernale, incanali tutta la tua forza nella distruzione della cagliata sinuosa. È stato l’odio eccessivo o solo sfortuna? Chissà - resta il fatto, la tua arma rimbalza e ti colpisce in testa, infliggendo metà del danno che volevi al formaggio. Sta ridendo di te? Sei sicuro di sentirlo ridere...
    \item Di che diavolo è fatta questa cosa? La tua arma si frantuma in centinaia di minuscoli pezzi. (Ti sembra un po' duro? - sai quanto possiamo affezionarci alle nostre armi preferite, se ti spezzerà il cuore del tuo combattente perdere la sua amata lama, vai avanti e lancia ancora se vuoi...)
    \item Mentre miri il tuo colpo e guardi minacciosamente il formaggio, senti un'aura di terrore che emana dalla sua superficie mite e cremosa. Le tue ginocchia iniziano a tremare. Le tue mani iniziano a tremare. Tu sudi. Tu tremi. E infine sprofondi in una pietosa pozza di gelatina debole per terra, piangendo di terrore e implorando i tuoi amici di portare via il brutto formaggio
\end{itemize}
\textbf{Attacco Magico}
\begin{itemize}
    \item In guardia! Abbassatevi! Sta tornando! Il tuo incantesimo rimbalza sul formaggio e torna dritto per colpirti (e chiunque sia abbastanza vicino a te da risentirne, ovviamente)
    \item Ooops - ehm, scusa? Il tuo incantesimo rimbalza selvaggiamente sul formaggio, colpendo un altro PG o NPG a caso.
    \item Hm, questa bacchetta funziona? Agiti il tuo strumento magico, lo scuoti, gli urli contro, lo minacci violentemente e rivolgi delle imprecazioni... eppure è impassibile, non ti aiuterà a lanciare nessun incantesimo contro questo povero formaggio innocente.
    \item Sei certo di aver lanciato qualcosa di spettacolarmente devastante... ma il tuo incantesimo inonda semplicemente il formaggio di una leggera e innocua spolverata di glitter argentati.
    \item Ti rimbocchi le maniche e guardi il formaggio severamente negli occhi...almeno lo faresti...se il formaggio avesse gli occhi ma mentre cerchi nella tua mente le parole del tuo incantesimo, ti colpisce quanto sia molto nobile il formaggio, semplicemente seduto lì com'è molto pacifico e mite, mite e tenero e invece del tuo incantesimo accuratamente memorizzato, parole di poesia fuoriescono dalle tue labbra, lodando la bellezza e lo splendore di questa meravigliosa cagliata. Rimani in questo modo per i prossimi dieci minuti, lacrime di emozione ti sgorgano negli occhi (a meno che qualcuno non ti schiaffeggi o ti versi dell'acqua addosso o qualcosa del genere).
    \item Quando inizi a lanciare il tuo incantesimo, c'è un minaccioso rombo di tuono sopra la testa, il cielo si oscura, il terreno trema, un gatto nero balza all'improvviso sul tuo cammino, qualsiasi specchio d'acqua vicino si trasforma in sangue e inizia a bollire, i destrieri del gruppo s’imbizzarriscono e cercano di mangiarsi l'un l'altro, la morte arriva su un cavallo pallido e un flusso costante di altri presagi minacciosi ti bombarda finché non cadi in ginocchio, rannicchiato nella paura e nella disperazione, il tuo gruppo terrorizzato ti implora e ti supplica di fermarti prima che tu possa provocare l'apocalisse.
\end{itemize}


\section{Capitolo 3}
Attualmente, il gruppo incontra una coppia di speziali itineranti-venditori di zuppe che non possono dire loro nulla del formaggio ma hanno molte cose interessanti in vendita e un sacco di zuppa calda.
Ai farmacisti piace sperimentare e se si siedono a tavola, scoprono che gli piace aggiungere alcune delle loro pozioni alla loro cucina. Se il gruppo tenta accidentalmente di pagare ai farmacisti zuppe o pozioni con il denaro mimic, i farmacisti sono furiosi e credono che i PG l'abbiano fatto per scherzo.
Se tutto va bene, però, si offrono di comprare il formaggio.
Se il gruppo accetta di venderlo, scoprono a pochi chilometri che il formaggio è tornato con loro (i Chwinga preferiscono la loro compagnia a quella dei farmacisti) e ora hanno due commesse arrabbiate che lentamente li pedinano determinate a prendere il loro formaggio o indietro i loro soldi.

\paragraph{Xanna and Melda Goldwood}
Elfo, femmina, mercanti
Xanna è un'astuta commessa che ha sempre gli occhi aperti per un buon affare, un affare o la possibilità di vendere qualcosa. Quasi tutto quello che esce dalla sua bocca è un tentativo di procurarsi beni o oro. Parla molto velocemente e, sebbene possa essere generosa con il suo tempo, si aspetta sempre una ricompensa in una forma o nell'altra.
Melda è stravagante, frivola e un po' sbadata. Ama sperimentare ma sicurezza e sensibilità non sembrano mai in cima alla sua lista di priorità. Le sue dita e i suoi vestiti sono macchiati e bruciati dal suo lavoro con le pozioni e canta allegramente tra sé e sé mentre vari odori ed esplosioni emanano dai suoi calderoni e fiasche coniche.
Melda è di natura generosa e darebbe via allegramente tutto ciò che possiede se Xanna non mantenesse un controllo molto stretto sulle cose.

\paragraph{Scena} \textit{Dietro la curva successiva si vede un carro allegramente dipinto sul ciglio della strada. Profumi deliziosi riempiono l'aria e dall'interno del carro si sentono due donne che litigano. “Ora dove hai nascosto il sale questa volta Melda? Eh? Rispondi! La scorsa settimana era in una bottiglia marchiata Thessaltossina!”
“Ehm? Cos'è che vuoi Xanna?" Tre bellissimi cavalli colorati pascolano pigramente sul ciglio dell'erba e sottili spire blu di fumo si alzano pigramente da un fuoco di cucina incustodito. "Mi stai ascoltando o hai di nuovo la testa tra le nuvole?"
"Oh cara, non posso assolutamente ascoltare tutto quello che dici, dolce Xanna, non smetti mai di parlare, ora passami quella bottiglia verde, vorresti?"
“Dice, PERICOLO ESPLOSIVO...”
"Oh, fidati, andrà tutto bene, voglio vedere cosa succede quando io..."
C'è un forte BANG e del fumo dall'odore nocivo si alza dal retro del carro mentre le due donne escono incespicando soffocando e sputacchiando. I tre cavalli alzano brevemente la testa e tornano al loro sgranocchiare, ovviamente senza allarmarsi.}
\textbf{Tabella 1} 1d6 :
\begin{enumerate}
    \item Filtro d'amore
    \item Pozione di Guarigione
    \item Pozione di volontariamente
    \item Niente - L'esplosione di Melda ha danneggiato il loro carro e distrutto tutte le loro scorte. Per fortuna nessuno è stato gravemente danneggiato.
    \item Pozione di chiaroveggenza
    \item Pozione dell'eroismo
\end{enumerate}
\textbf{Tabella 2}1d6: 
\begin{enumerate}
    \item Filtro d'amore
    \item Pozione della forma gassosa
    \item Pozione del Eroismo
    \item Pozione della Lettura della Mente
    \item Niente - Zuppa normale
    \item Pozione Guarigione
\end{enumerate}

\section{Capitolo 4}
Dopo aver lasciato le speziali, il formaggio continua ad affliggere il gruppo.
Se pensavano di essersene sbarazzati dalle farmaciste, si rendono presto conto che è tornato ed è di buon umore - rima, canta, lusinga e pungola i membri del gruppo a caso, cambiando da uno zaino a un altro e generalmente essendo irritante. Inoltre, inizia a emanare un odore piuttosto terribile.
Attualmente uno del gruppo si accorge che un grosso segugio nero dagli occhi di fuoco li sta seguendo a distanza.
Puoi far scattare il segugio sul tuo gruppo all'improvviso o farlo intravedere un po' per creare tensione: fai in modo che la persona di guardia di notte veda gli occhi infuocati che appaiono nell'oscurità, per poi svanire senza lasciare traccia. Il giorno dopo il gruppo potrebbe sentire i suoi ululati lugubri in lontananza... e poi avvicinarsi... finalmente scorgono finalmente l'enorme segugio nero su un tratto di strada solitario, che avanza piano verso di loro, scodinzolando felicemente.
È un segugio infernale (BR). Mantiene le distanze e si nasconde se il gruppo agisce in modo aggressivo, ma se il gruppo non è ostile diventa lentamente più audace e se sono amichevoli nei suoi confronti diventerà piuttosto l'animale domestico.
Il segugio infernale è interessato al Chwinga, non al formaggio, e quindi non mangerà la cagliata ma ci annuserà intorno e si attaccherà come colla a chiunque ce l'abbia nello zaino.
Il segugio infernale e il Chwinga sono stati compagni di viaggio in passato e il segugio era felicissimo di sentire ancora una volta il loro odore.
Il segugio infernale non si impegnerà in battaglia. Se il gruppo lo minaccia si ritira abbastanza da tenersi al sicuro poi riprende a seguirlo ancora una volta.
Se il gruppo viene attaccato, il segugio afferra il formaggio e si ritira con esso a distanza di sicurezza fino alla fine del combattimento, quindi lo restituisce scodinzolando.

\section{Capitolo 5}
Il gruppo si trova difronte ad un'ampia valle boscosa. La magia nel bosco è soppressa, ma i PG non possono saperlo finché non provano a lanciare un incantesimo. \\
\textbf{Incantesimi lanciati prima di entrare nella foresta entrerà in vigore quando ne usciranno}\\
\textbf{Le abilità magiche innate dei mostri restano in vigore, ma loro non sono in grado di usare comunque la magia}\\
La foresta si estende per 50km e ci vogliono da 1 a 3 giorni per oltreppassare la foresta.
\textbf{IL FORMAGGIO}: L'anello bardico e le magie protettive sono disabilitate, il segugio infernale prende il formaggio in bocca e lo protegge.

\paragraph{Anatre Sinistre} La foresta è popolata da una tribù di anatre sinistre. Queste anatre sono Toccate dall'Ombra (TCOE) - ad un certo punto sono entrate nell'Ombra e sono tornate... cambiate... La potente magia dell'Ombra non è influenzata da piccoli incantesimi mortali come il campo anti-magia. Ogni papera può quindi lanciare Invisibilità e un incantesimo di primo livello della scuola di Negromanzia o Illusione prima che abbia bisogno di un lungo riposo. Mangiare un'anatra Toccata dall'Ombra fa sì che il consumatore diventi Toccato dall'Ombra.
Le statistiche dell'anatra sono le seguenti: CA 11, PF 5, SAG 18, INT 20, FOR 2, DES 12, COS 8,
CAR 20.
Le anatre sono capaci di essere benevole e spaventose. Se un membro del gruppo scende a zero punti ferita mentre si trova nel bosco, un'anatra scivolerà con grazia giù dal baldacchino e toccherà la persona morente con il becco, conferendogli \textbf{Risparmiare i morenti} (BR) prima di svolazzare di nuovo nelle nebbie della foresta.
\paragraph{Eventi nel bosco}
\begin{enumerate}
    \item mostri - 1 Sciame di vespe 2 Ettercap 3 Un lupo mannaro
    \item NPG - Un goblin morto giace in mezzo al sentiero. Piccole impronte palmate circondano il corpo. Un controllo investigativo con CD 5 rivela che è stata beccato a morte da delle papere.
    \item Condizione del terreno : Un enorme albero caduto - uno di quegli antichi giganti della foresta - blocca completamente il sentiero. La sua mole aumenta di 25 piedi di diametro e per scalarla è necessario superare una prova di atletica con CD15 a meno che il PG non abbia attrezzatura da arrampicata.
    \item Papere sinistre :
        \begin{enumerate}
            \item     Un paio di piccoli occhi dorati ti fissano dalle profondità del fogliame scuro nelle vicinanze. Il suono di una campana dolorosa riempie per un momento l'aria intorno a te. Devi riuscire un tiro salvezza su Saggezza o subire 1d8 danni necrotici. Se sei già ferito, subisci 1d12 danni necrotici. Mentre vacilla per lo shock di quello che è appena successo, un'anatra nera con gli occhi dorati prende il volo dal cespuglio e vola via con grazia.
            \item All'improvviso, c'è un'anatra. Sai che non c'era nessuna anatra lì prima. Ma ora, indiscutibilmente, c'è un'anatra. La sua apparizione improvvisa ti disturba. Il suo sguardo indagatore penetra nelle profondità della tua anima. Balbetti, scuoti e incespica lontano da esso. Devi riuscire un tiro salvezza su Saggezza o diventare Spaventato dall'anatra fintanto che è nella tua linea di vista o 1d6 min.
        \end{enumerate}
\end{enumerate}

\paragraph{Incontro con le druide} Ad un certo punto durante il loro viaggio attraverso la foresta, il tuo gruppo sente la voce di una giovane donna che chiede aiuto. Indipendentemente dal fatto che decidano o meno di indagare, il loro percorso li porta in una palude dove possono vedere chiaramente una bellissima giovane donna umana appiccicata fino alle ascelle nel fango trasudante. Altre due belle giovani donne stanno su un terreno più solido vicino a una vecchia capanna sgangherata, tentando di afferrarla con rami e manici di scopa e piangendo disperatamente perché qualcuno venga ad aiutarle. Quando individuano il gruppo, lo implorano di assistere il loro amico che sicuramente annegherà se lasciato molto più a lungo!
Salvare Fleur richiede una prova di forza con CD15 o un po' di lavoro di squadra creativo a discrezione del DM.
Se il gruppo attacca le donne, queste sono inorridite e si ritirano all'interno della capanna. Quindi tornano armati di archi lunghi. Tutte e tre le donne usano il blocco delle statistiche da Arciere (VGTM) che è riassunto sopra.
Se il gruppo li aiuta, offrono assistenza in ogni modo possibile. Non possono fare nulla per il formaggio - li lascia perplessi
- ma possono offrire cure e consigli su come placare le papere (che a quanto pare sono particolarmente ghiotte di pomodorini).
\subparagraph{DRUIDE: Fleur, Fawn and Rowan} Umano, Druidi, Legale Bene
Tutti e tre i druidi usano il blocco delle statistiche Arciere, Volo's Guide to Monsters, riassunto di seguito:
AC16, HP75
Arco lungo: Gittata 150-600 piedi, +6 per colpire, 1 bersaglio, Colpire - 8 (1d8+4) danni perforanti.
Multiattacco: l'arciere effettua due attacchi con il suo arco lungo.
Occhio dell'arciere: 3/giorno: come azione bonus l'arciere può aggiungere 1d10 al suo prossimo attacco o tiro per i danni.
Le donne sono tre bravi druidi che si prendono cura delle creature e delle piante del bosco, anche se non sono completamente a loro agio con la recente invasione delle anatre. Sono loro che hanno lanciato il campo antimagia intorno alla foresta, ma l'interno della loro capanna è una zona di esenzione, che consente loro di curare e curare eventuali viaggiatori o animali selvatici feriti. Al momento ci sono due nani e un assortimento di animali selvatici che si stanno riprendendo in letti e gabbie all'interno della capanna.
Fleur è la donna caduta nella palude e le sue amiche Fawn e Rowan stanno cercando di aiutarla. Per quanto vogliano disperatamente aiutare il loro amico, nessuno di loro concederà di sollevare il campo dell'antimagia e permettere loro di usare la loro magia per farlo. Hanno fatto un giuramento quando si sono trasferiti qui e hanno intenzione di mantenerlo!
\section{Capitolo 6 }
Usciti dal bosco, il gruppo giunge quasi subito ad un accogliente monastero dedicato a Melissae, dea locale delle api.
I monaci sono estremamente amichevoli e insistono nel mostrare ai personaggi il luogo, dando loro tutto ciò di cui hanno bisogno per il loro viaggio, offrendo un pasto e, mentre stanno per partire, chiedono in cambio una piccola (grande) donazione.
Infatti, il gruppo troverà impossibile partire senza fare una cospicua donazione e/o acquistare parte del proprio filato casalingo articoli legati alle api dal “piccolo negozio” È improbabile che il segugio infernale voglia entrare nel monastero, si lamenterà e si metterà la coda tra le gambe, preferendo sgattaiolare via nei boschi e ricongiungersi al gruppo quando se ne andranno.

\paragraph{Scena}\textit{Mentre barcolli, inzuppato, prosciugato e probabilmente squilibrato dalle profondità del maledetto bosco infestato dalle anatre, le nuvole si aprono sopra di te e la tranquilla strada davanti a te è bagnata dalla mite luce del sole dorato. Pochi passi davanti a te c'è un monastero di pietra, le cui vetrate colorate brillano allegramente nella luce che scorre. Un monaco umano dall'aspetto allegro siede sui gradini suonando la sua arpa mentre un piccolo numero di api, uccelli e farfalle saltellano allegramente intorno alla sua testa calva. Ti saluta quando ti vede e grida parole di benvenuto: "Ave, viandanti! Mio Dio, sembri che potresti fare con una pausa e un po' di ristoro! Ecco, vieni dentro, vero? Abbiamo uno stabilimento balneare e confortevoli camere per gli ospiti, esperti di guarigione e un'ottima cucina casalinga, tutto a tua disposizione se ne hai bisogno!}
Il monaco è Fratello Andrenidae. Se il gruppo accetta la sua offerta, vengono accompagnati negli alloggi degli ospiti che sono caldi, puliti e accoglienti e hanno accesso agli stabilimenti balneari, alla sala da pranzo e all'infermeria se ne hanno bisogno.
(Se il gruppo decide di declinare l'offerta, Andrenidae fa loro una lezione sui pericoli della strada da percorrere, la sacralità delle api, le buone opere del monastero e la triste povertà che sopportano nel tentativo di convincerli a comprare qualcosa dal mercatino prima che se ne vadano. Se decidono ancora di andare avanti, vai direttamente a \hyperlink{fuco}{Il fascino del Fuco Fico...})
Se decidono di rimanere, una volta che si sono sistemati, il fratello Andrenidae li cerca e insiste affinché faccia loro una visita guidata del monastero.
Li conduce nei seguenti luoghi dove possono incontrare una varietà di NPG.
I monaci saranno insultati se i personaggi cercheranno di vendere loro il formaggio e incitati alla battaglia se il formaggio li insulta o verranno pagati con il denaro mimic per errore.
\paragraph{Monaci}
\begin{itemize}
    \item CA 16, HP 60, Velocità 13 metri
    \item Armatura dello sciame: il monaco evoca uno stemma d'api che conferisce +3 alla sua classe di armatura.
    \item Colpo senz'armi: attacco in mischia, portata 1,5 metri, +5 per colpire, 1 bersaglio. Colpito: 7 (1d8+3) danni contundenti.
    \item Il Monaco sceglie anche una delle seguenti abilità:
    \begin{itemize}
        \item Pungi come un'ape: il bersaglio deve superare un tiro salvezza di COS con CD13 o essere stordito fino alla fine del round successivo del monaco.
        \item Mente alveare: il bersaglio deve superare un tiro salvezza su WIS con CD13 o diventa Spaventato fino alla fine del round successivo del monaco.
        \item Trappola di Miele: il bersaglio deve superare un tiro salvezza di INT con CD13 o essere immobilizzato fino alla fine del round successivo del monaco.
    \end{itemize}
\end{itemize}

\subsection{Ambienti}
    \paragraph{Libreria}
    Suor Osmia può essere trovata qui mentre impartisce una lezione ai membri più giovani del monastero sul folklore e la mitologia delle api, in particolare la magica Ape delle nevi argentate, ritenuta ora estinta. Ha diversi esemplari conservati di api comuni in teche di vetro.
La biblioteca è una bella stanza dalle volte alte, piena di raggi di sole dalle enormi finestre ad arco. Ci sono migliaia di libri e tutti riguardano in qualche modo la dea Melissae o le sue api
    \subparagraph{Suor Osmia}Suor Osmia è una giovane monaca gnomo che ha scelto di dedicare la sua vita a Melissae dopo che un'ape le ha salvato la vita quando era piccola. È un po' confusa riguardo ai dettagli, ma non ha dubbi sul fatto che l'ape l'abbia salvata.
    \paragraph{Cucine}Le cucine sono pulite e luminose ma completamente disorganizzate e piene di un guazzabuglio di bottiglie e barattoli raggruppati su tavoli e scaffali colorati.
    Fratello Lasioglo si trova qui a fare vasetti di miele d'arancia speziato. Invita i PG a dargli una mano e spiega il processo per aggiungere fette di arancia, radice di zenzero e cannella ai vasetti prima di ricoprire di miele dorato in quantità inutili di dettagli, rendendo il processo assurdamente semplice laborioso e complicato. La sua vista debole non fa che aumentare la complessità poiché spesso raccoglie gli ingredienti sbagliati e deve iniziare da capo
    \subparagraph{Fratello Lasioglo}Le cucine sono pulite e luminose ma completamente disorganizzate e piene di un guazzabuglio di bottiglie e barattoli raggruppati su tavoli e scaffali colorati.
    Fratello Lasioglo si trova qui a fare vasetti di miele d'arancia speziato. Invita i PG a dargli una mano e spiega il processo per aggiungere fette di arancia, radice di zenzero e cannella ai vasetti prima di ricoprire di miele dorato in quantità inutili di dettagli, rendendo il processo assurdamente semplice laborioso e complicato. La sua vista debole non fa che aumentare la complessità poiché spesso raccoglie gli ingredienti sbagliati e deve iniziare da capo
    \paragraph{alveari}
    Fuori gli alveari si estendono per 5 km quadrate di prati fioriti.
Il fratello Halictus può essere trovato qui mentre controlla lo stato di salute di ogni colonia. Parla con entusiasmo delle diverse specie di api, dei loro modelli di comportamento e dei punti di interesse generali. Conosce ogni ape per nome e può dire a colpo d'occhio se una è malata, eccitata o di cattivo umore.
    \subparagraph{Fratello Halictus}Il fratello Halictus ha subito un trauma cranico da bambino che ha avuto un effetto pronunciato sul suo comportamento. Ritenuto "irraggiungibile" da ogni tutore che i suoi genitori benestanti potessero trovare, la sua famiglia lo portò al monastero disperata sperando che i monaci potessero "curarlo". Divenne presto evidente che l'entusiasmo, l'energia e l'arguzia del fratello Halictus, uniti alla sua tendenza a interessarsi profondamente a un singolo argomento, lo rendevano idealmente adatto alla vita monastica e divenne presto uno dei membri più informati e stimati della comunità.
    \paragraph{Catacombe}
    Le catacombe sono il luogo in cui i monaci seppelliscono i loro morti e come tali sono sacre e non visitabili. Frate Andrinedea mostra al gruppo l'ingresso e accenna al fatto che contiene i corpi dei cari defunti e dei loro tesori, poi prosegue rapidamente con il resto del tour.
Chi decide di avventurarsi per primo attraversa una serie di cantine di stoccaggio contenenti le provviste dei monaci e depositi di vestiario e cibo ecc.
Poi ci sono due caveau chiusi che contengono il denaro che i monaci accumulano per finanziare le loro buone opere.
Infine c'è una porta di legno chiusa che conduce alle catacombe, che consistono in tre strati di tunnel che scendono nelle profondità sotto il monastero
    \subparagraph{Primo piano}
    Consiste di cinque passaggi interconnessi fiancheggiati da scaffali contenenti bare di pietra. Qui è dove sono alloggiati i morti più recenti e la maggior parte delle bare di pietra sono in realtà vuote. Non ci sono trappole esplosive a questo livello poiché i monaci lo visitano ogni anno per accendere candele e pregare per i loro fratelli e sorelle defunti. I passaggi sono alti 2,5 metri per 1,5 metri di larghezza e torce spente stanno in applique lungo le pareti ogni 3 metri.
    Se i PG decidono di aprire le bare tirate 1d6 per un evento casuale:
    \begin{enumerate}
        \item La bara è vuota.
        \item Una vittima di peste in decomposizione recentemente deceduta. Lancia un d6, se ottieni un 5 hai la peste. Dopo 24 ore, si sviluppano febbre, foruncoli, ghiandole gonfie, occhi iniettati di sangue e un'irrefrenabile voglia di vomitare. Inizi anche a perdere 2d6 HP all'ora finché non guarisci dalla malattia. Chiunque sia stato a meno di 30 cm da te deve lanciare 1d6 per vedere se anche loro l'hanno contratta.
        \item Uno scheletro che stringe una sola rosa. Stranamente, la rosa è ancora in fiore.
        \item La bara è vuota.
        \item Uno scheletro che indossa una catena d'oro con un'ape conservata in una perla d'ambra che pende da essa.
        \item La bara è vuota.
    \end{enumerate}
    \subparagraph{Secondo Piano}
    La porta di legno a questo livello non è chiusa a chiave, ma una piastra a pressione si trova sotto la terza lastra all'interno. Richiede una prova di Saggezza (Percezione) con CD20 per individuarlo.
La piastra a pressione attiva una serie di finte bare da aprire su un sistema di cronometraggio a 2 minuti di distanza. Il Guardiano dei Terreni Furgliss è uno gnomo dall'immaginazione morbosa e i suoi grotteschi automi escono dalle loro bare nella cripta non illuminata con un aspetto abbastanza convincente, ma una prova di Investigazione con CD15 o una prova di percezione con CD18 riveleranno che in realtà sono falsi. Riduci la CD se il gruppo porta torce, ha una visione oscura o usa qualsiasi altra forma di luce. Gli automi assomigliano a vari orrori diversi ma hanno tutti le stesse statistiche. Prendono di mira l'essere vivente più vicino e combattono finché non vengono distrutti...\\

    \textbf{Automi Terrificanti} : CA 8, HPs22, Velocità 9 m
    ST 13 (+1), DEX 6 (-2), CON 16 (+3), INT 3 (-4) WIS 3 (-4), CAR 5 (-3)\\
    Immunità: Veleno, Psichico, affascinato, esausto, spaventato.\\
    \\Attacco: l'automa effettua un goffo colpo sulla creatura vivente più vicina. Attacco in mischia: +3 per colpire, raggiungere 5 piedi, un bersaglio. Colpo: 4 (1d6+1) danni contundenti
    \\
    Ci sono 4 bare con trappole esplosive che si aprono automaticamente a 2 minuti di distanza.
    Per determinare quale orrore sorge da quale bara, tira 1d6.
    
\begin{enumerate}
    \item Scheletro.
    \item Vampiro.
    \item Oooops, come ci è arrivato qui dentro?! C’è un vero Ghoul.
    \item Zombie.
    \item Wraith.
    \item Il meccanismo è rotto e la bara si apre solo parzialmente. L'automa ronza e sussulta cercando ripetutamente di uscire prima di rompere il suo meccanismo e cadere metà senza vita dentro e metà fuori dalla bara. Uno sguardo più attento rivelerà la verità della farsa.
\end{enumerate}

Ci sono 12 bare autentiche quaggiù e 6 casse chiuse piene di manufatti sacri e tesori. Scorri la tabella qui sotto per vedere cosa contiene ogni forziere o aggiungi i tuoi tesori se preferisci. Vengono dati sei artefatti ma puoi svuotare alcuni forzieri se preferisci.
\textbf{Manufatti religiosi nei forzieri.}\\
Tira 1d6 o scegli e mescola. Se i monaci trovano qualcuno di questi manufatti sui PG, tenteranno di confiscarli con ogni mezzo necessario.

\begin{enumerate}
    \item VASETTO DI MIELE DEL CONTENIMENTO \\
    Questo bellissimo vasetto di miele è realizzato in delicata porcellana color ambra a forma di alveare e decorato con delicate api cinesi sui lati e sul coperchio. Pesa 1 kg indipendentemente dal suo contenuto. All'interno c'è uno spazio extra dimensionale che può contenere fino a 660 kg di roba. Il miele, per esempio. A differenza di un sacchetto di contenimento, se il Vasetto di miele viene rotto tutto il contenuto viene immediatamente catapultato fuori da esso. Attualmente, il vaso di miele dell'azienda contiene l'ultimo sciame di api della neve d'argento esistente nel loro stato di ibernazione centenaria, unico nella sua specie.
    
    \item COGLIMIELE DEL COMANDO DELLE API \\
    Questo utensile in legno intagliato sembra innocuo, nonostante le incisioni decorative di api e fiori sull'asta. Tuttavia è un'arma potente, in grado di evocare 1d10 sciami di api (usa Sciame Di Insetti (Vespe) nelle Regole Base) per combattere al tuo fianco. Le api comunicano con te attraverso una danza interpretativa che puoi emulare e comprendere fintanto che tieni in mano il coglimiele.
    
    \item FLAGELLO DELLE API \\
    L'impugnatura di questa frusta incantata a cinque code è realizzata in ambra liscia, imbottita con velluto rosso per il comfort di chi la impugna. Ciascuna delle cinque code di cuoio intrecciate è intrecciata con cinque api viventi. Gittata 1,5 metri, 5d4 danni perforanti.
    
    \item PAPPA REALE \\
    Questa piccola latta rettangolare contiene un unico cubetto di gelatina color ambra che profuma deliziosamente di miele. Non devi assolutamente mangiarlo. Una volta che la gelatina viene fatta cadere a terra, si espande per riempire uno spazio di 1,5x1,5x1,5 metri. Ogni volta che inghiotte una creatura vivente si espande, fino a diventare un cubo che occupa uno spazio di 6x6x6 metri. Una volta che il cubo raggiunge la sua dimensione massima, si divide in 4 cubi più piccoli di 1,5x1,5x1,5 metri ciascuno. La pappa reale cerca il dominio del mondo. Comunica telepaticamente, incanalando le personalità delle creature che assorbe.
    
    \item Vuota.
    
    \item APE DELLA NEVE D'ARGENTO \\
    Questa bellissima ape regina color argento è conservata all'interno di un grande globo color ambra.
\end{enumerate}

    \textbf{PAPPA REALE.}\\
    CA 6, PS 84, Velocità 4.5, ST 14 (+2), DEX 3 (14),
    CON 20 (+5), INT 1 (-5), WIS 6(-2), CHA 1 (-5)
    Immunità: accecato, sordo, esaurimento, prono.
    Azioni: Nel suo turno, la pappa reale sposta la sua velocità (4.5 m). Qualsiasi creatura sul suo cammino deve effettuare una prova di destrezza con CD15 o essere inghiottita dalla gelatina. Con un risultato riuscito, la creatura può scegliere qualsiasi spazio adiacente in cui saltare per evitare la gelatina che avanza.
    Le creature avvolte all'interno della gelatina sono trattenute, non possono respirare e subiscono 6d6 danni da veleno all'inizio di ogni turno della gelatina.
    È necessaria una prova di forza con CD10 per sfuggire alla Pappa Reale o per salvare una creatura intrappolata all'interno.
    \\
    \textbf{APi DELLA NEVE D'ARGENTO}
    Queste api magiche non pungono, né producono miele. Possono lanciare in modo innato Raggio di gelo e Cono di freddo (BR).
Possono anche controllare il tempo entro un raggio di 1.5 m facendolo iniziare a nevicare o cessare e scongelarsi.
Le api delle nevi argentate si nutrono di fiocchi di neve e li trasformano in un siero blu traslucido che poi si indurisce in una sostanza cristallina in grado di assorbire la magia. Usano questo cristallo per costruire intricati alveari labirintici che assorbono la magia dall'area circostante.
La leggenda narra che l'ape regina d'argento sia in grado di usare due poteri unici per aiutare la sua colonia o anche altre creature che considera amiche:
\\Cent'anni di sonno: la regina emette un respiro gelido che congela tutto ciò che tocca in un perfetto riposo pacifico per 100 anni.
\\Ali di Guarigione: L'ape regina si libra avanti e indietro su una creatura ferita e una brezza gelida aleggia su di loro ripristinando 4d4+4 HP e facendola rabbrividire un po'.
Si sa poco di più su questi mitici insetti, ma i libri nella Libreria contengono le informazioni di cui sopra e gli stessi monaci potrebbero saperne ancora di più a discrezione del DM.

    \subparagraph{Terzo piano}
    Questo livello è un ossario, le cui pareti e il soffitto sono fatti di teschi e ossa, intervallati da affreschi scrostati che raffigurano nuvole di api macabre, ognuna con una minuscola testa di teschio, che attaccano il monastero. Altre nuvole di api escono dalle bocche di coloro che giacciono morti per terra. Sembra che i monaci li respingano con torce fiammeggianti.
Nessuno si è avventurato così lontano per secoli.
I sei monaci sepolti quaggiù furono tutti uccisi da un'antica e misteriosa piaga di api zombie.
Ogni cadavere è ora un nido in letargo di creature inquietanti. Se qualche bara viene aperta quaggiù, le api zombi sciamano fuori e attaccano. Sono immuni a quasi tutto tranne che ai danni da freddo e al fuoco. Qualsiasi ammontare di danni da freddo li rimanderà in letargo, ma possono essere veramente distrutti solo dal fuoco.
    \textbf{Sciame di Api Zombie}:\\
    CA 12, PS 22, Velocità di volo 9 m,
STR 3 (-4), DEX 13 (+1), CON 10 (+0),
CHAR 13 (+1)
Azioni: Pungiglione: Attacco in mischia: +3 per colpire, A contatto, un bersaglio nello spazio dello sciame, Colpire 10 (4d4) danni perforanti + 5 danni necrotici continui.
Chiunque venga ucciso da uno sciame di api zombie diventa un nido con un nuovo sciame dopo 2 minuti.
\\
Ad un certo punto durante l'indagine sulle catacombe, è probabile che i monaci sentano qualsiasi rumore che il gruppo sta facendo laggiù e allertino gli anziani monastici che verranno a indagare.
I DM possono aspettare il momento opportuno per portare i monaci, o lasciare che un tiro di dado decida ogni volta che i PG fanno una quantità significativa di rumore
\\
Quando gli Anziani entrano nelle catacombe e individuano gli avventurieri, le loro menti sospettose saltano immediatamente alla folle supposizione che il gruppo stia rubando le tombe. Tentano di dimensionare il gruppo e mandano corridori per la milizia locale che ha una torre di avvistamento a solo un miglio lungo la strada.
\\

\begin{itemize}
    \item Se il gruppo viene in silenzio o i monaci riescono a catturarli, i soldati di guardia arrivano con l'ora. Danno loro un segno di spunta, li costringono a fare una grossa donazione al monastero e li mandano a fare i bagagli.
    \item Se il gruppo cerca di persuadere gli anziani che non stanno facendo nulla di male, è necessaria una prova di persuasione con CD 15 per convincerli.
    \item Se decidono di combattere per la loro libertà, gli anziani sono pronti a morire per proteggere il loro monastero e 1d4 monaci sentono il trambusto e si uniscono alla mischia.
\end{itemize}

\paragraph{Il Negozietto}
A causa del sistema a senso unico che opera all'interno del monastero, è impossibile partire senza passare per il negozietto. Fratello Hylaeus è al banchetto e cerca di convicere i PG. 
\subparagraph{Fratello Hylaeus} I PG lo troveranno affascinante, premuroso e molto insistente, comprano qualcosa prima di andare... a meno che non siano arruffati e ricoperti dal sangue di altre persone, nel qual caso vedrà cosa può guadagnare dalla situazione.
Il fratello Hylaeus è un ladro ricercato che si nasconde nel monastero perché ha una taglia sulla testa. Si è guadagnato la fiducia degli altri monaci fingendo di essersi pentito dei suoi modi malvagi, ma in realtà detesta il posto e sta semplicemente aspettando il suo momento finché non sarà sicuro di andarsene. È un abile mercenario ed è abile nel manipolare le persone in circostanze disperate. Se può guadagnare qualcosa dai guai del partito, lo farà. Non è interessato a comprare il formaggio

\textbf{Articoli: }
\begin{itemize}
    \item \textbf{SAPONE AL MIELE - 20MO} \\
    Pulisce le cose! Inoltre - disinfetta e cura le ferite minori ripristinando 2d4 + 2 HP se fai il bagno con questo sapone. Per 24 ore dopo il lavaggio con questo delizioso sapone sentirai un profumo divino come un delizioso miele e guadagnerai +1 al tuo Carisma. Il sapone ha una durata di un mese (o 4 utilizzi) prima che si esaurisca.
    
    \item \textbf{BALSAMO CURATIVO AL MIELE - 200MO} \\
    Viene fornito in una piccola scatolina che contiene un po' di unguento sufficiente per un'applicazione ma ripristina 10d4 +20 HP.
    
    \item \textbf{TINTURA AL MIELE - 50MO} \\
    Bere una dose per curare il paziente da qualsiasi malattia, maledizione o veleno. Contiene una dose (ovviamente).
    
    \item \textbf{IDROMELE DI MELISSAE - 300MO} \\
    Questo delizioso idromele non ha solo un ottimo sapore, ma ti garantisce +5 a qualunque sia il tuo prossimo tiro dopo averlo bevuto. La bottiglia contiene abbastanza per quattro bicchieri. Non è consigliabile bere più di un bicchiere al giorno: un secondo bicchiere ti metterà fuori combattimento per ben 8 ore se non superi un tiro salvezza su COS 15 CD.
    
    \item \textbf{GREMBIULE API OPERAIE - 100MO} \\
    Bellissimo grembiule ricamato raffigurante fiori di campo e api. Triplica la velocità di chi lo indossa nello svolgere i lavori domestici. Il grembiule può essere indossato in sicurezza per 1 ora al giorno, altrimenti chi lo indossa soffre di 1 livello di esaurimento ogni ora.
    
    \item \textbf{CIABATTE SOFFICI DEL SONNO VIGILE - 50MO} \\
    Queste comode pantofole a forma di calabrone tengono i piedi al caldo. Ti permettono anche di sonnecchiare mentre stai facendo la guardia. Le ciabatte utilizzano 15 + la tua saggezza (percezione) per notare il pericolo in avvicinamento e svegliarti vibrando silenziosamente se sentono, vedono o percepiscono qualcosa nelle vicinanze.
    
    \item \textbf{PORTA SILENZIOSA - 50MO} \\
    Una minuscola porta di legno delle dimensioni di un palmo con incisa un'ape, confezionata in un sacchetto di velluto con una chiave d'argento. Posiziona la porta su qualsiasi superficie, metti la chiave nella serratura, ti ridurrai alla dimensione giusta per passare e tornerai alla normalità dall'altra parte.
    
    \item \textbf{TOVAGLIOLO SOUVENIR ESCAPOLOGICO - 10MO} \\
    Non andare mai da nessuna parte senza il tuo tovagliolo. Ha api ricamate. E può aiutarti a scappare. Afferra un'estremità e lancialo su un appiglio vicino fino a 18 metri di distanza. Lo strofinaccio si estenderà e si attaccherà all’appiglio. Può quindi essere scalato come una corda. Lo strofinaccio può sostenere fino a 1.360 kg di peso.
    
    \item \textbf{MIELE AROMATIZZATO ALL'ARANCIA - 1MO} \\
    Un regalo fantastico per la tua nonnina.
\end{itemize}

\subsection{Il Fascino del Fuco Fico...} Non importa come il gruppo riesca a lasciare il monastero, un'ape regina si è attaccata a uno dei PG senza che se ne accorgessero.
Lentamente nell'ora successiva attira api finché il suddetto PG non indossa un innocuo ma altamente impraticabile mantello di api.
Le api non danneggeranno il personaggio a meno che non siano provocate.
La soluzione ovvia a questo problema è rimuovere delicatamente l'ape regina (prova di abilità nella gestione degli animali / natura CD10) e metterla in un luogo sicuro, dopodiché le altre api la seguiranno.
Il tuo gruppo potrebbe avere altre idee.
Se i monaci non stanno già inseguendo il gruppo, lo saranno una volta che si renderanno conto che alcune delle loro api sono scomparse.

\section{Capitolo 7}
Mentre il gruppo continua a viaggiare, il formaggio continua a tormentarli.
La strada dal monastero si snoda attraverso aperti terreni agricoli e dolci colline. Il tempo è piacevole, gli uccelli cantano, ci sono piccoli ruscelli gorgoglianti di acqua fresca e limpida e il segugio infernale si è unito a loro ancora una volta.
Se vengono seguiti da qualsiasi NPG arrabbiato, ora potrebbe essere un buon momento per lasciarli recuperare e avere uno scontro di massa. Oppure puoi scegliere di aspettare un po', dipende interamente da te come DM.
Verso il tramonto, il gruppo si imbatte in un gruppo di bambini di diverse età che giocano a girotondo in un prato.
Tutto sembra sereno e i bambini chiedono ai PG di unirsi a loro.
Se il gruppo si unisce, tutto va bene per qualche breve istante finché il sole tramonta... e i bambini si trasformano in pietre simili a dolmen. Tutti i giocatori che si tengono ancora per mano con un bambino hanno la mano intrappolata nella pietra.
Se i personaggi scelgono di non unirsi al gioco, 2 streghe verdi (BR), che sono state in agguato nel boschetto di un bosco ai margini del campo, si travestono da belle e giovani fanciulle elfe e si uniscono invece.
Una volta che i bambini si trasformano in pietra le streghe sono intrappolate ma mantengono le loro forme illusorie e chiedono aiuto. Se il gruppo viene in loro aiuto, le megere le attaccano una volta che sono libere.
I bambini non sono reali. Questo antico cerchio di pietre è stato creato secoli fa da un vampiro morto da tempo a cui piaceva attirare bellissimi giovani mortali e intrappolarli qui in modo da avere un pubblico prigioniero per ascoltare la sua terribile poesia autoindulgente. Un controllo di intelligenza (storia) DC15 consentirà a un PG di ricordare questo pezzo di curiosità affascinante - e non del tutto inutile, se avete letto i libri nella libreria del monastero il DC scende a 10.
Stando così le cose, nessun bambino piccolo verrà danneggiato danneggiando le pietre erette.
Hanno AC17, 30 HP per 7 cm quadrati. Ogni pietra è alta 4 metri e spessa 1 metro. Una creatura la cui mano è bloccata probabilmente ha bisogno di distruggere 7-30 cm di pietra per liberarsi, a seconda delle dimensioni della loro mano e di quanto è bloccata. Se qualcuno avesse acquistato un tovagliolo escapologico dal negozietto, potrebbe tornare utile a questo punto, riuscendo velocemente a scappare

\section{Capitolo 8}
Una volta che il gruppo è finalmente arrivato a destinazione, (ovvero: qualunque città o villaggio tu debba raggiungere) se provano a consegnare il formaggio scoprono presto che non c'è Château Roquefort, non c'è mai stato.
Si recano alla Taverna più vicina e cercano ospitalità per la notte.
Tuttavia, Larrs Thurbald, il padrone di casa della Trota Sballata, si interessa molto al formaggio e gli offre un ottimo prezzo.
I Chwinga sono ora felici di lasciare il formaggio ed esplorare il loro nuovo ambiente. Consenti a personaggi adeguatamente abili una prova con CD media per individuare il Chwinga che lascia il formaggio.
A discrezione del DM, possono anche rivelarsi al gruppo e ringraziarlo per la loro compagnia tramite Benedizione con qualcosa dal Tabella Benedizione di Chwinga o regalando loro l'anello bardico.
In alternativa, potresti anche farli benedire dal Chwinga prima che partano e i personaggi scopriranno i doni misteriosi e non sapranno mai da dove sono venuti.
Ad ogni modo, il padrone di casa taglia e serve il formaggio e tutti lo dichiarano delizioso.
Un intenditore locale lo identifica come la cagliata vincitrice del concorso di quest'anno - del valore di diverse migliaia di pezzi d'oro - che è stata rubata poche settimane fa. Mentre tutti gli sguardi si girano sospettosi verso il povero e innocente padrone di casa, sta al gruppo se saltare in sua difesa con la loro storia di improbabile verità o sgattaiolare fuori dal retro e lasciare rapidamente la città...


\section{Benedizione Chwinga}
\begin{enumerate}
    \item Benedizione del fuco amico: Hai vantaggio sui tiri di carisma quando si tenta di fare amicizia con qualcuno o di guadagnarne la fiducia.
    \item Benedizione dell’ape cercatrice: Hai vantaggio su investigazione e percezione ogni volta che cerchi tesoro.
    \item Benedizione della lingua melliflua: Hai vantaggio su carisma ogni volta che si usa una storia, una canzone o una danza interpretativa per cercare di influenzare qualcuno.
    \item Benedizione del formaggiologo: Hai una profonda conoscenza di tutto ciò che riguarda il formaggio. Sei in grado di collegare quasi tutto sotto il sole a qualche interessante curiosità sul formaggio.
    \item Benedizione dell’ape operaia: La tua operosità diventa quasi sovrumana e sei in grado di svolgere compiti al doppio della velocità che faresti normalmente, fino a un massimo di cinque volte al giorno.
    \item Benedizione dell’alveare infernale: Le vespe infernali ti considerano un amico e non ti attaccheranno. Potrebbero persino portarti dei regali. Che tu voglia il tipo di regali che portano le vespe infernali è un'altra questione...
\end{enumerate}

\end{document}
